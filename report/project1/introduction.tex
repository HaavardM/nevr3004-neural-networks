\section{Introduction}
Any single neuron only contain very little information, but networks of many interconnected neruons can represent complex information. Head direction cells (HD) in the brains of mice are no exception, and by combining small pieces of information from each neuron it is possible to reconstruct the head direction of mice.

Predicting the true head direction may be compared to the work of a detective in the CSI shows on television. The detectives are rarely able to find the suspect after collecting one single piece of evidence, but usually need use multiple pieces to build a case. Each piece new piece of evidence may help answer simple yes/no questions, helping the detectives in their search. In the beginning all citizens could be a suspect, but the detectives can narrow the search as they collect more evidence. In the brain of mice, any single neuron only contain limited information about the head direction, but by using multiple neurons as "evidence" the brain is still able to create a good estimate of the current head direction downstream.  

By investigating the information in a set of HD cells, we will try to understand how the information is distributed across different cells, as well as exploring how the brain may combine the information from multiple cells to percieve the current head direction.
We will also analyze whether a change of basis in the data may help us visualize patterns in the data. While considering each neuron independently during data collection is natural, it may not be the best option when analyzing the data later on. We will try to create a new basis using PCA and simplify data analysis to further understand the relation between HD cells and head direction. By applying polar coordinates transformation, we are able to uncover a circular dependency.

The aim of this paper is to investigate the information hidden in HD cells, and to determine a relationship between the spike-train of multiple cells and the head direction.

