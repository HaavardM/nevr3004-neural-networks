\section{Introduction}
Any single neuron only contain very little information, but networks of many interconnected neruons can represent complex information. Head direction cells (HD) in the brains of mice are no exception, and by combining small pieces of information from each neuron it is possible to reconstruct the head direction of mice.


By investigating the information in a set of HD cells, we will try to understand how the information is distributed across different cells, as well as exploring how the brain may combine the information from multiple cells to percieve the current head direction.
We will also analyze whether a change of basis in the data may help us visualize patterns. While considering each neuron independently during data collection is natural, it may not be the best option when analyzing the data later on. We will try to create a new basis using PCA and simplify data analysis to further understand the relation between HD cells and head direction. By applying polar coordinates transformation, we are able to uncover a circular dependency.

The aim of this paper is to investigate the information hidden in HD cells, and to determine a relationship between the spike-train of multiple cells and the head direction.

