\section{Discussion}
The aim of this paper were to uncover the information in the spike-train of head direction neurons in the brain.
We found that the neurons contains useful information about the head direction of mice. By creating tuning curves and calculating the mutual information between head direction and neural activity, we have postulated how the neurons can be used to update the set of possible head direction, much the same way a detective use evidence to limit the search for a suspect. By only knowing the spike-train of multiple HD cells, we can make edjucated guesses about the true head direction.

Further, by applying a change of basis using PCA and by utlizing the first few principal components we have found a circular relationship between the head direction and the first two principal components.  This could be used to develop good models for for the head direction, using simple statistical learning methods. 

While we are able to create edjuacted guesses, we can only approach the problem of predicting head direction using a probabilistic approach. Each bit of information we gain help us shape the probability distribution over different head directions, but we never gain a definitive answer. 


\cite{projectdata} have used the same dataset, and created online estimators using bayesian methods. By using the activity of different cells, they were able to use the data to update a prior belief (the likelihood of different head directions) and create good posterior distributions. Their experiments also showed similar activity during sleep. This appears to be a working proof-of-concept, indicating it is in fact possible to predict head direction from neural activity. Their experiments also provide experimental evidence of the postulated ring-attractor hypotheis, where theoretical models assume the HD cells lies on a virtual ring. This hypoethesis fits well with our findings on the circular dependencies between the first two principal components and the head direction.  

Further research into the field may uncover better models for predicting head direction, which may be used to gain more insight into how our brain process information. It may also lead to better human-machine interfaces in the future, by allowing machines to predict and understand our senses. 