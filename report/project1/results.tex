\section{Results}

\subsection{Convincing tuning curves}
\begin{figure}[H]
    \centering
    \begin{subfigure}[b]{0.4\textwidth}
        \includegraphics[width=\textwidth]{figs/Mouse12-120806_awakedata.mat/T3C10.eps}
        \caption{Cell with convining tuning curve. There appears to be a strong reaction to stimuli.}
        \label{fig:1a_high_mut_info}
    \end{subfigure}
    \begin{subfigure}[b]{0.4\textwidth}
        \includegraphics[width=\textwidth]{figs/Mouse12-120806_awakedata.mat/T2C2.eps}
        \caption{Cell with unconvincing tuning curve, and the stimuli appears to be uncorrelated with the response.}
        \label{fig:1a_low_mut_info}
    \end{subfigure}
    \caption{Two cells from mouse 12 with different tuning curves}
    \label{fig:1a_two_tuning_curves}
\end{figure}
To understand how different cells reacts to different stimuli, we created tuning curves from \cref{eq:firing_rate}. Some of the cells showed very convincing tuning properties with strong peaks for specific stimuli with little background, such as in \cref{sub@fig:1a_high_mut_info}. Others appears to mostly be noise, and does not appear to contain much information such as shown in \cref{fig:1a_low_mut_info}. 
\begin{figure}[H]
    \centering
    \includegraphics[scale=0.8]{figs/1b.eps}
    \caption{Promising tuning curves from two different mice and different brain areas. }
    \label{fig:1b_convincing_tuning}
\end{figure} 
In \cref{fig:1b_convincing_tuning} there are 4 different neurons from different mice and brain areas, with different convincing tuning properties. Neuron T4C14 also showes how the tuning curve wraps around when the stimuli goes beyond $360^\circ$. Most of the neurons appears to have unique tuning properties, and tend to prefer different stimulus. By combining multiple neurons it appears to be possible to at least perform an edjucated guess about the current head direction by knowing the activitity in different HD cells. The different tuning curve spikes appear to be well distributed around a circle, i.e. the cells seems to cover most head directions well. 
\subsection{Peculiar tuning properties}
\begin{figure}[H]
    \centering
    \includegraphics[scale=0.8]{figs/peculiar_tuning.eps}
    \caption{Cell with peculiar tuning properties. Notice how the cells have multiple firing patterns.}
    \label{fig:peculiar_tuning}
\end{figure}
Some cells also had more peculiar tuning properties with multiple firing patterns. \cref{fig:peculiar_tuning} shows three cells with multiple firing patterns, but which still appears to contain useful information about head direction. These cells may not indicate one specific head angle, but by knowing when these cells are active we can still use the information to rule out other possibilites and shape the distribution of possible head directions.

\subsection{Mutual information}
To quantify the amount of information about head direction for each cell, the mutual information score were calculated using \cref{eq:mutinfo_disc}. Each of the plots include the mutual information score in the title, and from \cref{fig:1a_two_tuning_curves} we can see how the mutual information tend to be higher for the more convincing tuning curves. The mutual information score is in bits (shannons) per unit time, and can the thought of as how many yes/no answers (bits) we gain on average after observing a cell during a sufficiently short time interval \cite{mutualinfo}.

\begin{figure}[H]
    \centering
    \begin{subfigure}[b]{0.49\textwidth}
        \includegraphics[width=\textwidth]{figs/weird_mutual.eps}
        \caption{Convincing tuning curves with low mutual information}
        \label{fig:weird_mutual}
    \end{subfigure}
    \begin{subfigure}[b]{0.49\textwidth}
        \includegraphics[width=\textwidth]{figs/probs.eps}
        \caption{Prior probability distribution}
        \label{fig:probs}
    \end{subfigure}
\end{figure}

However, low mutual information score does however not imply that the corresponding tuning curve is of no interest. In \cref{fig:weird_mutual} there are three tuning curves which looks convincing, but which results in a low mutual information score. By also looking at the prior probability density function in \cref{fig:probs}, we can see how these tuning curves have their peak in a head direction which was less visited by the mice. The prior indicate what we believed the head angle distribution were before observing any cell activity. When observing cell activity which contradicts our prior belief (i.e. high activity in cells which triggers at an unlikely head direction), we simply need more information to "accept it". Observing high activity in these neruons contradicts our prior belief and would make us more uncertain about what the true head direction really is (i.e. all possibilites becomes more equally likely). 

\subsection{Principal component analysis}
\begin{figure}[H]
    \centering
    \begin{subfigure}[b]{0.49\textwidth}
        \includegraphics[width=\textwidth]{figs/Mouse12-120806_awakedata.mat/pca.eps}
        \caption{PCA for mouse 12}
    \end{subfigure}
    \begin{subfigure}[b]{0.49\textwidth}
        \includegraphics[width=\textwidth]{figs/Mouse28-140313_awakedata.mat/pca.eps}
        \caption{PCA for mouse 28}

    \end{subfigure}
    \caption{Principal components analysis for the two datasets. The principal components are visualized both in cartesian coordinates as well as in polar coordinates. The plots clearly shows similar stimuli clustering together. The polar coordinate angle $\theta$ appears to explain the difference head direction well.}
    \label{fig:pca}
\end{figure}
To better visualize and utilize the data, PCA analysis were used to reduce the number of variables, while still keep as much information as possible. In the rather beautiful plot \cref{fig:pca} it shows the two first principal components, where the scores are color-coded according to the head direction. The plots clearly shows how the different stimuli is grouped by color, and there appears to be a strong relationship between the principal components and the head direction. The two components explains about $15\%$ of the variance which considering the amount of cells (variables in the original dataset) makes this a very useful visualization. 
It also appears to be a slight trend in the samples, where the colour appears to get brighter with clockwise motion. By applying a non-linear transformation to polar coordinates, we see how the head direction is well explained by the rotation $\theta$ (not to be confused with the head direction), and only barely depends on the radius $\rho$. Notice how the color increase in intensity almost linearly as principal component scores rotate around the circle. This indicates a circular relationship between the first two principal components and the head direction. 
