\section{Results}
To interpret the response of the different cells to stimuli, tuning curves were created by calculating and plotting \cref{eq:firing_rate} agains the stimuli (head direction). We can see from \cref{fig:1a_two_tuning_curves} how different the responses to the stimuli may be for different cells. Some cells may have a obvious correlation between stimuli and response, while others appears to more or less unaffected.
\begin{figure}
    \centering
    \begin{subfigure}[b]{0.4\textwidth}
        \includegraphics[width=\textwidth]{figs/Mouse12-120806_awakedata.mat/T3C10.eps}
        \caption{Cell with convining tuning curve. There appears to be a strong reaction to stimuli.}
    \end{subfigure}
    \begin{subfigure}[b]{0.4\textwidth}
        \includegraphics[width=\textwidth]{figs/Mouse12-120806_awakedata.mat/T2C2.eps}
        \caption{Cell with unconvincing tuning curve, and the stimuli appears to be uncorrelated with the response.}
    \end{subfigure}
    \caption{Two cells from mouse 12 with different tuning curves}
    \label{fig:1a_two_tuning_curves}
\end{figure}
\begin{figure}[H]
    \centering
    \includegraphics[scale=0.8]{figs/1b.eps}
    \caption{Promising tuning curves from two different mice and different brain areas. }
    \label{fig:1b_convincing_tuning}
\end{figure}

