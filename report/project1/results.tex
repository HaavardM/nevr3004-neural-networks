\section{Results}
To test which of the neurons are reacting to the head direction, tuning curves were created from \cref{eq:firing_rate}. The tuning curves show how the mean firing rate varies with stimulus, and give a good indication of which cells contains information about the head direction and which do not. \cref{fig:1a_two_tuning_curves} are examples of two very different tuning curves which shows how much the response varies for different cells. 
In \cref{fig:1b_convincing_tuning} there are tuning curves from two cells in each session, and it shows how there appears to be a clear reaction to certian stimuli for some of the cells.

To quantify the amount of information the spike-train of a cell contains about the head direction, the mutual information was calculated using \cref{eq:mutinfo_disc}. The information is quantified as bits per unit time, and tells us the number of bits of information we expect to get during a short period of time for a given cell. \cref{fig:1b_convincing_tuning} also includes the corresponding mutual information for two different cells.

\begin{figure}
    \centering
    \begin{subfigure}[b]{0.4\textwidth}
        \includegraphics[width=\textwidth]{figs/Mouse12-120806_awakedata.mat/T3C10.eps}
        \caption{Cell with convining tuning curve. There appears to be a strong reaction to stimuli.}
        \label{fig:1a_high_mut_info}
    \end{subfigure}
    \begin{subfigure}[b]{0.4\textwidth}
        \includegraphics[width=\textwidth]{figs/Mouse12-120806_awakedata.mat/T2C2.eps}
        \caption{Cell with unconvincing tuning curve, and the stimuli appears to be uncorrelated with the response.}
    \end{subfigure}
    \caption{Two cells from mouse 12 with different tuning curves}
    \label{fig:1a_two_tuning_curves}
\end{figure}
\begin{figure}[H]
    \centering
    \includegraphics[scale=0.8]{figs/1b.eps}
    \caption{Promising tuning curves from two different mice and different brain areas. }
    \label{fig:1b_convincing_tuning}
\end{figure}

