\section{Introduction}
The Hopfield Network is a simple model for artifical neural networks (ANN). Compared to more modern ANNs, the Hopfield network do not have the same predictive abilities, but it serves as an important basis of modern ANNs. By building complex networks of interconnected neurons, the Hopfield network is able to store and remember multiple patterns, and therefore also serves as an importan model for understanding the mechanism behind associative memory in humans and animals. While the Hopfield network may not be the most used in the field of ANNs, there are practical uses of the 

As humans we have the ability to recall memories from partial information, such as from incomplete images or by observing similar patterns. An apple is an apple, independent of its color or excact shape. The Hopfield Network may help us understand how it is possible to either restore lost information from partial inputs, or how we can recall memories by association.

In this paper we will investigate the stability properties of the Hopfield network and show how storing memories creates local optimums where the network reach an equilibrium, allowing the network to restore multiple memories using a predefined set of network weights. 