\section{Introduction}
The Hopfield Network is a simple model for artifical neural networks (ANN). Compared to more modern ANNs, the Hopfield network do not have the same predictive abilities, but it serves as an important basis of modern ANNs. By building complex networks of interconnected neurons, the Hopfield network is able to store and remember multiple patterns, and therefore also serves as an importan model for understanding the mechanism behind associative memory in humans and animals. While the Hopfield network may not be the most used in the field of ANNs, it do exists use-cases such as image reconstruction. 

As humans we have the ability to recall memories from partial information, such as from incomplete images or by observing similar patterns. An apple is an apple, independent of its color or excact shape. The Hopfield Network may help us understand how it is possible to either restore lost information from partial inputs, or how we can recall memories by association.

By borrowing the concept of energy and work from physics, we will create a physical interpretation of the Hopfield network, and try to understand the dynamics and limitations of the Hopfield network.


In this paper we will investigate the dynamics and stability properties of the Hopfield network, using both numerical and analytical methods. We will show how storing memories creates local equilibriums where the network remains stable, allowing the network to restore multiple memories using a predefined set of network weights. Further we will show how each stored pattern coincide with two equilibrium points. We will also show how the Hopfield network may be used to reconstruct defective QR-codes.